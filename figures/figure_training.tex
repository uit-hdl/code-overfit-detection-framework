\documentclass{article}
\usepackage{tikz}
\usetikzlibrary{shapes.geometric, arrows, positioning, decorations.pathreplacing}

\begin{document}

\begin{figure}[ht]
\centering
\begin{tikzpicture}[node distance=2cm and 2cm, auto]

\foreach \x in {1,...,5}
{
  \node[yslant=0.5, draw] (wsi\x) at (\x,0) 
  {\includegraphics[width=.2\textwidth, height=.2\textheight]{./wsi\x.png}};
}

\draw[decorate, decoration={brace, amplitude=10pt, mirror}]
    (-1,-3) -- (7, -3) node [midway, below=10pt] {Training data}
  ;

    \coordinate (base) at (0,-10);

    \def\numLayers{5}
    \def\offset{0.2}

    
    % Draw layers
    \foreach \n in {1,...,\numLayers}
    {
        % Calculate shift for each rectangle
        \pgfmathsetmacro{\shiftX}{(\numLayers - \n) * \offset}
        \pgfmathsetmacro{\shiftY}{(\numLayers - \n) * \offset / 2}

        % Draw rectangle
        \draw[fill=blue!30, opacity=0.9] ([shift={(\shiftX, \shiftY)}]base) rectangle ++(.4,-3);
    }

    % Label
    \node at ([shift={(2*\offset, -\numLayers*\offset - 2.5)}]base) {Model / Feature Extraction};


% \foreach \x in {1,...,5}
% {
%   \node[yslant=0.5, draw] (wsi_labeled\x) at (\x,-10) 
%   {\includegraphics[width=.2\textwidth, height=.2\textheight]{./wsi\x.png}};
% }
% 
% \draw[->, thick] (3, -4.5) -- (3, -6); % Connects the midpoint below the brace to the position of the first image
% 


\end{tikzpicture}
\end{figure}

\end{document}

